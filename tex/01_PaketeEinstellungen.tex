% --- Pakete einbinden
% Bedingte Kompilation für verschiedene TeX-Engines
\usepackage{iftex}
% Silbentrennung nach ausgewählter Sprache (vgl. \documentclass)
\usepackage[english, main=ngerman, shorthands=off]{babel}
\ifLuaTeX
	% Schriftart-Einstellungen für LuaLaTeX
	\usepackage{fontspec}
\else
	% Verwendung der Zeichentabelle T1 (Sonderzeichen etc.)
	\usepackage[T1]{fontenc}
	% Legt die Zeichenkodierung fest, z.B. UTF-8
	\usepackage[utf8]{inputenc}
\fi
% Text-Schriftart
\usepackage{lmodern}
% „fancy“ Schriftart (\Fontlukas, \Fontamici)
\usepackage{aurical}

% Verbessertes Aussehen von Text
\usepackage[final]{microtype}
\ifLuaTeX
	% Verbesserte Behandlung von Ligaturen
	\usepackage{selnolig}
\fi
% Einstellung des Zeilenabstands
\usepackage{setspace}
% Alternative Implementation von \raggedright etc.
\usepackage{ragged2e}
% Automatische Anführungszeichen
\usepackage{csquotes}
% Formatierung von Telefonnummern
\usepackage{phonenumbers}

% Nutzen von „+“, „-“ etc. in \setlength, \setcounter, …
\usepackage{calc}
% Nützliche LaTeX-interne Befehle
\usepackage{ltxcmds}
% \ifthenelse-Befehl
\usepackage{xifthen}
% \forloop-Befehl
\usepackage{forloop}
% Optionen für eigene definierte Befehle
\usepackage{xparse}
% Befehle mit Optionen der Form „<key>=<value>, …“ definieren
\usepackage{keyval}

% Farben
\usepackage{xcolor}
% Zum Einbinden von Grafiken (\includegraphics)
\usepackage{graphicx}
% .tex-Dateien mit \includegraphics einbiden
\usepackage{gincltex}
% Bessere Verarbeitung von Dateinamen für \includegraphics etc.
\usepackage{grffile}
% Niemals „draft“ für graphicx verwenden
\PassOptionsToPackage{final}{graphicx}
% Abbildungen im Fließtext (\begin{wrapfigure}…)
\usepackage{wrapfig}
% Bilder in der Mitte von zweispaltigem Text
\usepackage[nomicrotype]{pullquote}
% Zeichnen/Diagramme in LaTeX
\usepackage{tikz}
% Zusätzliche Definitionen für absolute Positierung auf der Seite mit TikZ
\usepackage{tikzpagenodes}

% Mathepaket (Grenzen über/unter Integralzeichen)
\usepackage[intlimits]{amsmath}
% Symbole, \mathbb etc.
\usepackage{amssymb}
% Einige für Physik nützliche Mathe-Befehle
\usepackage[notrig]{physics}
% Chemische Formeln
\usepackage{chemformula}
% „Schöne“ Brüche im Fließtext mit \sfrac
\usepackage{xfrac}
% Ermöglicht die Nutzung von „\SI{Zahl}{Einheit}“
\usepackage{siunitx}

% Optionen für Listen
\usepackage{enumitem}
% Darstellen einer Seite im Querformat mit automatischer Drehung in
% der PDF-Ansicht (\begin{landscape}…)
\usepackage{pdflscape}
% Seiten für Notizen
\usepackage{notespages}
% Wird für Kopf- und Fußzeile benötigt
\usepackage[draft=false]{scrlayer-scrpage}
% Einstellen der Seitenränder
\usepackage[vtex=false, dvips=false, pdftex=false]{geometry}
% Mehrspaltiger Text
\usepackage{multicol}
% Ändern des Aussehens des Inhaltsverzeichnisses
\usepackage{etoc}

% Tabelle mit vorgegebener Breite und X-Spalten (passen sich automatisch an)
\usepackage{tabularx}
% Tabellenoptionen (z.B. vertikal zentrierte Paragraphen mit „m“)
\usepackage{array}
% \multirow-Befehl in Tabellen
\usepackage{multirow}
% gestrichelte Linien (z.B. \hdashline); nach tabularx laden?
\usepackage{arydshln}

\usepackage{diagbox}

% Verlinkt Textstellen im PDF-Dokument
\usepackage{hyperref}
% Erstellen von Glossaren; nach „hyperref“ laden!
\usepackage{glossaries}
% „Schlaue“ Referenzen; nach „hyperref“ laden!
\usepackage{cleveref}

% Text u.a. drehen (Hinzugefügt für \rotatebox in einem Vorstellungsrunden-Artikel)
\usepackage{rotating}

\usepackage{hhline}

% --- Einstellungen
% -- geometry (Seitenränder)
\geometry{
	left=1.1cm,
	right=1.1cm,
	top=0.3cm,
	bottom=0.35cm,
	headsep=0.1cm,
	headheight=1.3cm,
	footskip=1.1cm,
	includeheadfoot
}
% Für Schnittmarken und Randbereich
% Nur aktivieren, wenn \fibelcropmarks = true
\providecommand{\fibelcropmarks}{false}
\Ifstr{\fibelcropmarks}{true}{
	% crop-Paket benötigt luatex85 für Kompatibilität
	\usepackage{luatex85}
	% DIN A4: 210mm×297mm (d.h. 7mm Schnittrand an allen 4 Rändern)
	% Hinweis: Damit das crop-Paket richtig funktioniert, dürfen die Abmessungen
	% der Seite (Kopf, Fuß, Ränder etc.) ab hier nicht mehr geändert werden!
	\usepackage[cam, center, width=224truemm, height=311truemm]{crop}
}{}

% -- (La)TeX
% Mehr Freiraum in Tabellen
\renewcommand{\arraystretch}{1.2}
% „Einsame“ Zeilen (heißen anscheinend „Schusterjungen“ bzw. „Hurenkinder“) am
% Seitenanfang/-ende vermeiden (default: 150; max: 10000)
\clubpenalty=3500
\widowpenalty=3500
% Zusätzlichen vertikalen Freiraum der Umgebungen „center“, „flushleft“ und
% „flushright“ unterdrücken
\AtBeginEnvironment{center}{\topsep=0pt\partopsep=0pt}
\AtBeginEnvironment{flushleft}{\topsep=0pt\partopsep=0pt}
\AtBeginEnvironment{flushright}{\topsep=0pt\partopsep=0pt}

% -- koma
% Für Überschriften sollen keinerlei Nummern angezeigt werden.
% Trotzdem wird aber der Zähler „section“ benötigt, wenn z.B. ein Zähler mit
% jeder neuen \section zurückgesetzt werden soll.
% Setze also „secnumdepth“ so, dass \section als nummeriert gilt, deaktiviere
% aber die tatsächliche Ausgabe der Zahl über \sectionformat.
\setcounter{secnumdepth}{\sectionnumdepth}
\renewcommand{\sectionformat}{}
% Überschriften zentrieren
\let\raggedsection=\centering
% Abstände vor und nach \subsection und \subsubsection ändern
% KOMA-Skript default für \subsection und \subsubsection:
% beforeskip=-3.25ex plus -1ex minus -.2ex, afterskip=1.5ex plus .2ex
\newcommand{\fibelspacingsubsection}[1][subsection]{
	\RedeclareSectionCommand[
		beforeskip=-2.5ex plus -0.5ex minus -1.2ex,
		afterskip=1ex plus 0.2ex minus 0.5ex
	]{#1}
}
\newcommand{\fibelspacingsubsubsection}[1][subsubsection]{
	\RedeclareSectionCommand[
		beforeskip=-0.75ex plus -0.5ex minus -0.7ex,
		afterskip=1sp minus 1ex
	]{#1}
}
\fibelspacingsubsection
\fibelspacingsubsubsection
% Überschriften in Serifen
\addtokomafont{sectioning}{\rmfamily}
% \begin{description}… in Serifen
\addtokomafont{descriptionlabel}{\rmfamily}
% Kopf-/Fußzeile in normaler Schriftart
\addtokomafont{pageheadfoot}{\normalfont}
\addtokomafont{pagenumber}{\normalfont}
% Seitenzahl groß
\addtokomafont{pagenumber}{\LARGE}
% Kopf- und Fußzeile konfigurieren (eckige Klammern: \pagestyle{scrplain},
% geschwungene Klammern: \pagestyle{scrheadings})
% Kopf/Fuß zunächst vollständig leeren
\clearscrheadfoot
% Mitte der Kopfzeile
% … ungerade Seitenzahlen
\cohead[]{{\large Fachschaft Physik}\\\fibelheadsepline[odd]}
% … gerade Seitenzahlen
\cehead[]{{\large Universität Münster}\\\fibelheadsepline[even]}
% Mitte der Fußzeile
\cfoot[\pagemark]{\pagemark}

% -- graphicx
% Standardmäßig „keepaspectratio“ verwenden
% s. https://tex.stackexchange.com/a/91619/51235
\setkeys{Gin}{keepaspectratio}

% -- hyperref (Links/Verweise)
\hypersetup{
	unicode,
	% Links/Verweise mit Kasten der Dicke 0.5pt versehen
	pdfborder={0 0 0.5},
	% Links auch im Entwurfs-Modus („draft“) erzeugen
	final
}

% -- csquotes (Anführungszeichen)
% Anführungszeichen automatisch umwandeln
\MakeOuterQuote{"}

% -- phonenumbers (Telefonnummern)
\setphonenumbers{
	country=DE,
	foreign=international,
	area-code-sep=space
}

% -- tikz (Zeichnungen)
% Positionsberechnungen bei TikZ
\usetikzlibrary{calc}

% -- siunitx (Einheiten)
\sisetup{
	locale=DE,
	forbid-literal-units,
	mode=text,
	binary-units,
	quotient-mode=fraction,
	per-mode=fraction,
	fraction-function=\sfrac,
	detect-weight
}
\DeclareSIUnit{\euro}{€}
\DeclareSIUnit{\LP}{LP}

% -- cref (Verweise)
% Verweise auf Fußnoten als hochgestelltes Fußnotensymbol darstellen
\crefformat{footnote}{#2\textsuperscript{#1}#3}

% -- url
\makeatletter
% Keine Zeilenumbrüche bei Doppelpunkten in URLs (wegen https://…)
\g@addto@macro{\UrlNoBreaks}{\do\:}
\makeatother

% -- etoc (Inhaltsverzeichnis)
% Definition des Aussehens des Inhaltsverzeichnisses
\etocsettocstyle
% before
{\section*{Inhaltsverzeichnis}
\setlength{\parindent}{0cm}
\RaggedRight
% Warnungen im Inhaltsverzeichnis deaktivieren
\hbadness=10000
\vbadness=10000
\hfuzz=1cm
% Wegweiser im Inhaltsverzeichnis, wird von Text umflossen
\begin{pullquote}{shape=image, textcoldist=6mm, objvoffset=15, objdist=5mm, image=private/res/toc_wegweiser.png, imageopts={width=10.5cm}}}
% after
{\end{pullquote}
\cleardoublepage}

% Definition des Aussehens von sections im Inhaltsverzeichnis
\etocsetstyle{section}
% start (first section)
% s. auch https://tex.stackexchange.com/q/97782
{\advance\csname @rightskip\endcsname 1em
	\advance\rightskip 1em
	\advance\parfillskip -1em}
% prefix
{}
% content
{\normalsize\etocname~~\dotfill~\makebox[\rightskip][r]{\etocpage}\pullquotenl\par}
% end (last section)
{}

% Definition des Aussehens von \subsection, \subsubsection im Inhaltsverzeichnis:
% Nicht anzeigen ;-)
\etocsetstyle{subsection}{}{}{}{}
\etocsetstyle{subsubsection}{}{}{}{}

