\section*{\Huge\Fontlukas{Willkommen am Fachbereich Physik!}}
\begin{multicols*}{2}
Jetzt ist es soweit: Ihr fangt mit eurem Studium an!
Ab jetzt ist alles neu und ungewohnt, aufregend und anstrengend.

Damit es nicht ganz so schlimm wird, ist ja noch die Fachschaft da und mit ihr zusammen die Ersti-Fibel.
Mit diesem Heft sollt ihr durch die ersten schweren Wochen eures Studiums geleitet werden. Diese Zeitschrift wird jetzt schon seit vielen Jahren immer wieder von Freiwilligen zusammengeschrieben.
Die Redaktion besteht also nur aus Studenten, die Stunden ihrer Zeit opfern, damit ihr dies hier lesen könnt.

Die gesamte Redaktion wünscht euch viel Spaß beim Lesen und auch beim Studieren.
Wir freuen uns sehr, wenn der eine oder die andere zum Mitschreiben bei der nächsten Ausgabe motiviert werden kann. Auch Kritik ist jederzeit erwünscht.

Eure Redaktion der\\
Ersti-Fibel

\vspace{-0.75cm}
\hspace{2cm}
\includegraphics{res/fsphys_stempel.png}

\vspace{1.8cm}
\includegraphics[width=\columnwidth]{private/res/nichtlustig/021119_energie.jpg}
\vspace{\fill}

\columnbreak

% XXX Jedes Jahr das Jahr im Bild aktualisieren!
\includegraphics[width=\columnwidth]{private/res/comics/studis_zeitverlauf.pdf}

\vspace{-2ex}
\section*{Impressum}
\vspace{-1ex}

{\centering
	\vspace{-1ex}
	\minisec{\large\normalfont Redaktionsleitung:}
	\vspace{-1ex}
	\redaktionsleitung
	
	\footnotesize
	Layout: Simon May, Benedikt Bieringer, Marius Willer\\
	Cover: Andreas Gieselmann, Fernando Romahn
	
	Textsatz mit \LaTeX
	
	% XXX Jedes Jahr Druckerei & Auflage einfügen (falls geändert)!
	Druck: CCC Druck und Medien GmbH\\
	Coerdestr.\ 44, 48147 Münster\\
	1.~Auflage: 130~Stück

	\vspace{-1ex}
	\minisec{\large\normalfont Kontakt:}
	\vspace{-1ex}
	Fachschaftsrat Physik\\
	c/o Institut für Kernphysik\\
	Wilhelm-Klemm-Straße 9\\
	48149 Münster
	
	Tel.: \phonenumber{0251 83}[34985]\\
	E-Mail: \email{fsphys@uni-muenster.de}\\
	Web: \url{https://www.uni-muenster.de/Physik.FSPHYS}
\par}

\footnotesize
An dieser Stelle möchte sich die Redaktion bei allen bedanken, die zum Entstehungsprozess der diesjährigen Ersti-Fibel beigetragen haben.
%Besonders den ungenannten Korrektoren und Personen im Hintergrund wollen wir danken.
%
Unser Dank gilt auch allen, die sich schon im Voraus für die Ersti-Woche engagiert haben und denen, die dieses in der Woche selbst tun werden.
Ohne euch wäre dies alles gar nicht möglich.

% XXX Jedes Jahr Anzeigen/Sponsoren eintragen (natürlich vorher welche finden)!
Selbstverständlich wäre diese Zeitung ohne unsere Sponsoren der
\begin{center}
	UKM~Blutspende, UKM~Stammzellspende sowie der Techniker Krankenkasse
\end{center}
ebenfalls nicht möglich gewesen, daher hier noch einmal herzlichsten Dank für Ihre Unterstützung.
\end{multicols*}

