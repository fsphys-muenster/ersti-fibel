\section{Sorgen, Ängste, Nöte?}
\subsection{Wer hilft bei Fällen von Sexismus?}
\textbf{Ein unangemessener Spruch vom Dozenten wie "Das sollten sich die Damen im Raum noch einmal vertieft anschauen!" oder Kommentare von Komillitonen, die einem mehr oder weniger direkt weismachen wollen, dass eine Frau in den Naturwissenschaften sowieso nichts verloren und erst recht nichts zu gewinnen hat. Oder der ungewollte körperliche Kontakt auf einer Party. Solche Situationen kennen viele Frauen entweder von sich selbst oder haben sie als beobachtende Person miterlebt. Oft stellt sich bei den Betroffenen ein Gefühl von Hilfosigkeit und Scham ein, das Suchen und das Fragen nach Hilfe scheint voller Hürden.} 

\begin{multicols}{2}
Das alles sind Dinge, die man nicht erleben möchte – schon gar nicht zu Studienbeginn, wo man sich gerade eigentlich wirklich auf andere Dinge konzentrieren möchte und muss. Aber auch an der Universität gibt es (Macht-)Strukturen, die solche Situationen begünstigen. 
Bei Fällen von Machtmissbrauch, sexueller Übergriffigkeit und geschlechterbasierter Diskriminierung ist das Gleichstellungsbüro eine gute erste Anlaufstelle. Mir ist es ein Anliegen, mit euch eine Lösung für eure Probleme und Sorgen zu entwickeln, die in eurem Interesse ist oder einfach zuzuhören. 

Vielleicht ein kurzer Satz zu mir: Ich bin Franzi, 23 Jahre alt und studiere schon etwas länger an der Uni Münster; meine Fächer sind Latein und Judaistik. Seit einem Jahr bin ich außerdem Gleichstellungsbeauftragte für die Statusgruppe der Studierenden, also euch!

Das Gleichstellungsbüro kümmert sich aber nicht nur um die Fälle, die oben geschildert werden – zum Aufgabenbereich gehört unter anderem auch die Beratung bei Einstellungsverfahren in der Lehre oder die Förderung von jungen Frauen in den Wissenschaften – zum Beispiel als Kontrollinstanz bei der Vergabe von Stipendien. 
Zwar ist in der gesetzlichen Grundlage für die Aufgaben des Gleichstellungsbüros nur von Frauen die Rede, mir ist es aber ein großes Anliegen, Wege zu finden, wie auch nicht binären, trans, inter und agender Personen im Alltag geholfen werden kann. 

Diese Unterstützung kann verschiedene Formen annehmen. Mein momentanes Hauptanliegen ist das konsequente Vorgehen gegen sexualisierte Gewalt. Dazu gehört vor Allem, die verwirrenden Strukturen, die in solchen Fällen tätig werden (können), transparent aufzuschlüsseln und generell mehr Aufmerksamkeit für dieses oft totgeschwiegene Thema zu erzeugen. 
Aber auch das Studieren mit Kind birgt Herausforderungen, bei denen ihr Unterstützung bekommen könnt: Ich helfe euch gerne dabei, zum Beispiel einen KiTa-Platz oder eine Möglichkeit zur Betreuung in den „Randzeiten“ (also früh morgens oder spät abends) zu finden. Im Studi-Kidz-Café habt ihr außerdem die Möglichkeit, euch mit anderen Eltern, die gerade studieren, auszutauschen. 
%Während des Studienalltags kommt es häufig zu Situationen, an denen man einfach nicht weiterkommt: seien es Formulare, die Prüfungsvorbereitung oder der ganz normale Uni-Stress. Um diese Momente etwas einfacher zu machen, möchte ich in meiner Amtszeit als stellvertretende zentrale Gleichstellungsbeauftragte aus dem Bereich der Studierenden gerne ein Projekt anstoßen, bei dem Studis aus höheren Semestern den "Neuen" als Buddies, also ganz auf Augenhöhe und informell, zur Seite stehen. 
%Außerdem ist es eine Herzensangelegenheit von mir, alle Menschen, die sich als Frauen identifizieren, in meiner Arbeit mit einzubeziehen. 
%Diese Unterstützung kann ganz verschiedene Formen annehmen: Das Studieren mit Kind zum Beispiel birgt Herausforderungen, bei denen euch geholfen werden kann: Ich helfe euch gerne dabei, zum Beispiel einen KiTa-Platz oder eine Möglichkeit zur Nachmittagsbetreuung zu finden. Im Studi-Kidz-Café habt ihr außerdem die Möglichkeit, euch mit anderen Eltern, die gerade studieren, auszutauschen. 
Solltet ihr also ein Anliegen haben, bei dem ihr Unterstützung oder ein offenes Ohr braucht, schreibt mir gerne eine E-Mail an \email{studglei@uni-muenster.de}! 

Weitere Informationen findet ihr unter: 
\begin{center}
  \includegraphics[width=0.7\columnwidth]{res/StudzGB.png}
  \smallskip
  \url{https://www.uni-muenster.de/Gleichstellung/stud\_angebote.html} 
\end{center}

Habt einen guten Start ins Studium!
 
\fibelsig{Franzi}
 
\end{multicols}
