\section{Lernen \& Bücher an der Uni}
\begin{multicols*}{2}
\begin{quote}
	\textit{Ein Raum ohne Bücher ist wie ein Körper ohne Seele.}
	
	\hfill--- Cicero
\end{quote}
Schon früh im Studium wird klar werden: Ohne Lösen der Übungszettel werden auch die Klügsten am Physikstudium scheitern.
Und da das Rechnen für die Meisten einige unüberwindbare Hürden darstellt, ist ein Austausch unvermeidbar.
Die Zettel lassen sich am Besten gemeinsam lösen; alle haben andere Ideen und zusammen konvergiert dies bei den meisten Aufgaben gegen eine schöne Lösung.

Andere wiederum rechnen lieber hochkonzentriert alleine.
Insbesondere in höheren Semestern hat man oft mit individuellen Problemen zu kämpfen und auch für Klausuren empfiehlt es sich, manchmal alleine zu lernen.
Da dies nicht immer in der fetzigen neuen WG möglich ist, wo immer eine tolle Ablenkung parat ist, gibt es zum Lernen die entsprechenden Räume an der Uni.

\includegraphics[width=\columnwidth]{res/buecher_studibib.png}

Ein anderes großes Thema im Studium sind die richtigen Lehrbücher.
Jede Person hat hier einen unterschiedlichen Zugang.
Manchmal reicht das Skript der Profs allein, manchmal jedoch möchte man vielleicht darüber hinaus etwas erfahren oder die Theorie nochmal auf einem vielleicht anderen Weg nachvollziehen.
Im Studium werden viele Themen angerissen, die bei Interesse auch nochmal vertieft werden können.
In höheren Fachsemestern ist man dann oft so spezialisiert, dass man vielleicht in der Bibliothek der theoretischen Physik das passende, verstaubte Buch sucht.

Die Uni stellt natürlich für alle Bedürfnisse eine gute Infrastruktur zur Verfügung.

\includegraphics[width=\columnwidth]{res/buecher_studio.png}

\subsection{StudiO \& StudiBib}
Für gemeinsames Lernen und Diskutieren steht das StudiO ("für \textbf{Studi}erende \textbf{O}ffen") zur Verfügung; hier kann in Gruppen gearbeitet werden.
Das StudiO ist ein Raum zwischen der Kernphysik und der Angewandten~Physik, direkt vor dem Hörsaal und neben dem Lernzentrum.
Es stehen Whiteboards und ein PC zum Recherchieren im Internet zur Verfügung.
Das StudiO ist ein Kreativzentrum, ein wahrer "Thinktank".
Wenn man Probleme hat, Aufgaben nicht lösen kann, dann finden sich hier bestimmt Leute, die helfen können.
Am Whiteboard kann dann gemeinsam über die Aufgabe diskutiert werden.
Auch Studierende höherer Semester sind hier anzutreffen, welche bestimmt auch kurz Zeit für Erstis zum Klären einer Verständnisfrage haben.
Außerdem liegen die gängigsten Lehrbücher aus dem Besitz der Fachschaft (also nicht mitnehmen!) aus, in denen bei Problemstellungen oft nachgeschlagen werden kann.

Falls die Bücher nicht ausreichen oder die Atmosphäre im StudiO zu unruhig ist, gibt es noch die StudiBib (Studierenden-Bibliothek) im Erdgeschoss der IG1.
Hier standen früher die Bücher der Fachschaft zur Verfügung, doch seit der Einführung des Bachelorstudiengangs hat die Fachschaft leider keine Studierenden mehr gefunden, die freiwillig Präsenzdienst dort geben und die Bücher beaufsichtigen.
Wir haben aber die Integration der Literatur in das Lernzentrum gestartet, um diese wieder für Studierende zugänglich zu machen.
Sie umfasst eine große Auswahl von Standardlehrbüchern in Physik, Mathematik und Chemie.

\subsection{Seminarräume}
Falls es im StudiO und in der StudiBib jedoch keinen freien Platz zum Lernen gibt oder es einfach zu unruhig ist, so stehen insbesondere am Nachmittag viele Seminarräume leer.
Einfach mal auf die Belegungspläne an der Tür schauen und nichts wie rein!

\subsection{Lernzentrum}
Das sogenannte Lernzentrum ist die wichtigste Bibliothek für Studierende am Fachbereich.
Aus dem StudiO führt eine Tür direkt dorthin.
Auch hier kann man viele Standardlehrbücher und auch Standardwerke für höhere Semester finden.
Eigentlich ist das Lernzentrum die Bibliothek der Angewandten Physik~(AP).
Diese wird von SHK-Stellen beaufsichtigt und hat in der Regel die Öffnungszeiten:
\begin{itemize}
	\item Mo. - Fr. von 11:00~Uhr bis 17:00~Uhr 
\end{itemize}
Auch ein Multifunktionsgerät (Scanner/Kopierer) sowie Arbeits- und Computerplätze sind hier vorhanden.

\textbf{Aufgrund der Corona-Pandemie ist das Lernzentrum aktuell geschlossen, deshalb informiert ihr euch am Besten unter \url{https://www.uni-muenster.de/Physik.AP/Organisation/Bibliothek.html} über den aktuellsten Stand der Dinge und zukünftige Öffnungszeiten.}

\begin{center}
	\fibelimgtext{
		\includegraphics[width=.9\columnwidth]{res/xkcd/238_pet_peeve_114.png}
	}{\url{https://xkcd.com/238}}
\end{center}

\subsection{Weitere Bibliotheken am Fachbereich}
Auch die anderen Institute haben Bibliotheken.
So ist in der offiziellen Fachbereichsbibliothek im 5.~Stock der IG1 eine große Anzahl an Büchern zu finden.
Auch die theoretische Physik hat eine gut gefüllte Bibliothek im 3.~Stockwerk der TP mit Büchern aus dem Bereich der theoretischen Physik.
Hier findet sich so manch alter Schatz, der schon lange nicht mehr von den Verlagen aufgelegt wird.
Leider haben diese Bibliotheken und auch die anderen Institutsbibliotheken sehr begrenzte Öffnungszeiten, welche auf der Homepage des Fachbereiches nachgeschlagen werden können.

\fibelimgtext[below left]{
	\includegraphics[width=\columnwidth]{res/buecher_ulb_gehorche_keinem.jpg}
}{\parbox{\columnwidth}{Foto: Philip Brechler\hfill \url{https://www.flickr.com/photos/plaetzchen}}}

\subsection{Die ULB}
Das Angebot der ULB~(Universitäts- und Landesbibliothek) am Krummen~Timpen in der Nähe des Juridicums ist breit gefächert.
Hier wird nicht nur ein reichhaltiges Angebot an Büchern bereitgestellt, sondern es gibt auch große Bereiche zum Stillarbeiten und eine kleine Caféteria.
An der ULB ist auch die Lehrbuchsammlung angesiedelt; hier gibt es die gängigsten Standardlehrbücher diverser Fachrichtungen in großer Anzahl zum Ausleihen und mit nach Hause nehmen.
Im ULB-Online-Katalog oder über die Suche "disco" kann man die Standorte der Bücher finden.
So manch ein seltenes gebrauchtes Buch findet sich auch im Keller der ULB wieder, welcher ein unendliches Labyrinth von Bücherregalen ist.
Damit ihr Literatur ausleihen könnt, müsst ihr allerdings noch euren Studierendenausweis für die ULB aktivieren. Die Anleitung dazu und weiteres zu den Angeboten der ULB findet ihr unter \url{https://www.ulb.uni-muenster.de/}.

\fibelsig{Friedrich}

\begin{center}
	% \includegraphics[width=.4\columnwidth]{res/comics/be_rational_get_real.pdf}
	\includegraphics[width=.8\columnwidth]{res/comics/be_rational_get_real_wide.pdf}
\end{center}

\end{multicols*}
