\section[Das Sommerfest der Fachschaft]{Die Physik braucht ein Sommerfest!\\Das Sommerfest der Fachschaft}
\begin{multicols*}{2}
% Etwas mehr Abstand zwischen Paragraphen, weil der Artikel so kurz ist
\setlength{\parskip}{1.6\parskip}

\textbf{Was wäre ein Sommer ohne Sommerfest?
	Nicht auszudenken!
	Deswegen (und nicht etwa aus Freude am Feiern und Organisieren) veranstaltet die Fachschaft Physik dies seit Jahren.}

\includegraphics[width=\columnwidth]{res/sommerfest_grill.png}

Um euch einen Überblick geben zu können, was auch euch im Sommer des nächsten Jahres erwarten wird, hier der Bericht diesen Jahres:

Auch in diesem Jahr kam das Sommerfest der Fachschaft~Physik bei allen Beteiligten gut an und war ein großer Erfolg.
Insgesamt kamen ungefähr \num{3,2e2}~Leute (Zahl grob durch Autor geschätzt, kann von realen Gegebenheiten abweichen) und genossen das umfangreiche Rahmenprogramm bei Bier, Musik, Bratwürsten und Grillkäse.

Pünktlich um 11:00~Uhr wurde der Grill angezündet und so konnten die ersten Besucher als Mittagessen eine gute Bratwurst genießen.

Durch eine Fragestunde zum Erasmus-Programm mit ehemaligen Erasmus-Studenten und dem Erasmus-Beauftragten, Prof.\ Dr.\ Kohl, hatte der Tag auch eine fast schon bildende Komponente.

Die gleißende (aber eigentlich ganz angenehm warme) Sonne ließ die Teams des Volleyballturniers nicht davon abhalten, das lang ersehnte Spiel um 13:00~Uhr auszuführen, und so trafen die Mannschaften im K.\,O.-System aufeinander und boten sich einen harten, aber fairen Kampf um den Sieg.
Von der Vorgruppe über Viertel- und Halbfinale bis hin zum Finale konnte jedes Team sein Können unter Beweis stellen.

Auch in diesem Jahr wurde wieder ein Speed-Cubing-Wettbewerb veranstaltet, bei dem die Teilnehmer unter Beweis gestellt haben, dass sie den Zauberwürfel mühelos lösen können.

Um 16:15~Uhr wurde das von vielen herbeigefieberte Physiker-Duell veranstaltet.
Bei diesem nach dem Prinzip des alten "Familien-Duells" konzipierten Spiel trafen vier Arbeitsgruppen des Fachbereiches Physik aufeinander und stellten sich den Fragen der Moderatoren Marius und Fernando.

Zuvor wurden "100~Physiker" (Physikstudenten) gefragt und es galt natürlich, die meistgegebenen Antworten zu erraten.
In diesem Jahr traten die vier Arbeitsgruppen von Professor Doltsinis, Pernice, Heusler und Busse in einem unfassbar spannenden Duell gegeneinander an, bei dem schließlich die AG um Professor Busse den Sieg erlangte.

Unterlegt wurde der ganze Tag von Musik, die mit einem breiten Spektrum an musikalischen Meisterwerken eine perfekte Kulisse schuf.

Da der Abend nach hinten hin offen gehalten wurde, blieben viele noch bis zum bitteren Ende um ca.~23:00 Uhr und beteiligten sich sogar noch an den Aufräumarbeiten.

Ein langes und erfolgreiches Sommerfest war nun beendet und sucht im nächsten Jahr einen würdigen Nachfolger.
Dann natürlich auch mit euch, die ihr jetzt schon mal herzlichst zum größten, öffentlichen seiner Art am Fachbereich eingeladen seid.
Besonders für die verschiedenen Turniers werden natürlich weitere starke Gegner gesucht!

\fibelsig{Benedikt}

\medskip
\includegraphics[width=\columnwidth]{res/sommerfest_zelt.png}
\end{multicols*}
