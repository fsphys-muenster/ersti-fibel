\section[Gleichstellungsbüro am FB11]{Gleichstellung betrifft uns ALLE\\ Das Gleichstellungsbüro stellt sich vor}

\begin{multicols}{2}

\textbf{Auch das Gleichstellungsbüro des Fachbereichs möchte Euch herzlich an der \UniMuenster{} und am Fachbereich begrüßen.}

\begin{wrapfigure}[6]{r}{3.4cm}
    \vspace*{-0.3cm}
    \includegraphics[width=\linewidth]{res/gst_buero.jpg}
\end{wrapfigure}

Die ersten Veranstaltungen der O-Woche sind vorbei, alles ist neu. Der Hörsaal sieht ganz anders aus als ein Klassenzimmer und sicherlich habt ihr auch schon eure Mitstudierenden gesehen, die, genau wie ihr, gerade einen neuen Lebensabschnitt mit dem Studium der Physik beginnen. Doch habt ihr mal darauf geachtet, wie es bei der Verteilung der Geschlechter aussieht? Was habt ihr dabei festgestellt? Vermutlich seht ihr ziemlich viele männliche Personen und deutlich weniger andere Geschlechter vertreten. Untersuchungen gibt es dazu, dass die Anzahl der Physikerinnen entsprechend des Kaskadenprinzips mit steigenden Karrierestufen weiter abnimmt: Es gibt weniger Doktorandinnen als Studentinnen, und noch weniger weibliche Post-Docs und Professorinnen. Warum ist das so? Welches strukturelle Problem liegt dem zugrunde? Was können wir dagegen tun? Wer hilft mir, wenn ich sexualisierte Belästigung beobachte oder selbst erfahre? Wer unterstützt mich bei Fragen der Vereinbarkeit von Elternschaft und Studium? Für diese Fragen kannst du dich ans Gleichstellungsbüro wenden.

Das Gleichstellungsbüro agiert am Fachbereich in engem Austausch und Zusammenarbeit mit der dezentralen Gleichstellungsbeauftragten, Carina Bücker, und mit der Gleichstellungskommission. Wir organisieren Veranstaltungen zur Information, zum Austausch und zur gegenseitigen Unterstützung. Bei einigen Veranstaltungen heißen wir gerne alle Personen des Fachbereichs willkommen, andere bieten wir nur für ausgewählte Gruppen an, zum Beispiel nach Geschlecht oder Zugehörigkeit im Fachbereich.

Je nach Veranstaltung gibt es zum Beispiel einen wissenschaftlichen Input-Talk mit Ergebnissen und Erkenntnissen aus der Gender \& MINT Forschung. Dazu laden wir Personen für einen Vortrag ein, die in diesem Gebiet forschen. So bekommen alle die Möglichkeit, die eigenen Gender-Kompetenzen auszubauen - keine Angst, Vorkenntnisse werden nicht benötigt. Im Anschluss werden beispielsweise Fragen geklärt, der Vortrag und das Thema besprochen, aber auch gerne andere Themen aufgebracht und Erfahrungen geteilt.

Wenn im Allgemeinen Physikalischen Kolloquium eine Rednerin spricht, laden wir diese zu einem informellen Gespräch ein. Am selben Tag ihres Vortrags treffen wir uns vorab und sprechen sowohl über ihren Karriereweg als auch über geschlechterbezogene Herausforderungen, denen sie während ihres eigenen Werdegangs begegnet ist. Meistens haben diese schon weiter fortgeschrittenen Wissenschaftlerinnen super Tipps für Studentinnen.

Weiterhin kümmern wir uns um Prävention, Fragen und Erfahrungen von sexualisierter Diskriminierung, Belästigung und Gewalt. Wir haben immer ein offenes Ohr für euch, gehen mit euren Sorgen und Nöten diskret um und kennen die nötigen Ansprechpersonen, wenn weitere Schritte eingeleitet werden sollen. Bitte zögert nicht, euch zu melden, wenn ihr etwas beobachtet oder euch etwas passieren sollte.

\begin{multicols}{2}
Alle Infos dafür findet ihr auf unserer \mbox{Website\footnotemark}. Oder ihr lasst euch von uns regelmäßig Informationen über unseren Newsletter zuschicken. Meldet euch dafür ganz einfach über diesen QR-Code an:
\fibelimgtext{
    \includegraphics[height=4.5cm]{res/gleichstellung_qr.pdf}\\ %Newsletter QR-Code
}{\url{http://go.wwu.de/v-rti}} % https://listserv.uni-muenster.de/mailman/listinfo/physikerinnen-fb11
\end{multicols}
\footnotetext{{\url{www.uni-muenster.de/Physik/department/equality/}}}

Grundsätzlich gilt: Wir sind die ersten Ansprechpersonen, wenn es um Gleichstellungsthemen geht. Also ruft uns einfach an \mbox{(0251 83-33516)}, schreibt uns eine E-Mail (\textbf{\email{gleichstellung.physik@uni-muenster.de}}) oder kommt auf ein Gespräch vorbei (Raum: AP106) und wir helfen euch.


\fibelsig{Anna Niemann, Janice Bode, Barbara Leibrock}
\end{multicols}
