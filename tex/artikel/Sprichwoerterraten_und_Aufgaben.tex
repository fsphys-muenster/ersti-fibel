\section[Sprichwörterraten und Aufgaben zur praktischen Physik]{Sprichwörterraten\dots}
\vspace{-0.5cm}
\textbf{Welche Sprichwörter verbergen sich hinter diesen mehr oder weniger wissenschaftlich ausgedrückten Satzkonstruktionen?}

{\footnotesize
Die meist unerwünschten Fragmente eines transluzenten, anorganischen Werkstoffes geben der günstigen Fügung des Schicksals Geleit.

Mit Nutriment gefüllt exploriert es sich suboptimal.

Ein der Anziehungskraft unterliegendes liquides Kontinuum bildet in Sedimenten und Magmatiten konkave Vertiefungen.

Die Fläche hinter einem Körper, der für Licht undurchdringlich ist, verhält sich reziprok zur Strahlung.

Wer eine nicht für sich selbst bestimmte, quaderförmige Ausschachtung in der Pedosphäre vornimmt, wird sich deren Solenbereich unter Einfluss der Gravitation nähern.

(Lösungen stehen auf Seite \pageref{pg:rätsel_lösungen})\par}
\vspace{-0.8cm}
\fibelsig{Konrad}

\vspace{-0.5cm}
\section*{{\dots}und Aufgaben zur praktischen Physik}
\vspace{-0.6cm}
\textbf{Konstante: $\mathbf{m_\text{Kuh} = \SI{400}{\kg}}$}
\vspace{-0.4cm}

% Abstände bei \subsection reduzieren
\fibelspacingsubsubsection[subsection]

\subsection[Mechanik]{Mechanik:}
Eine Kuh galoppiere aus einer bestimmten Entfernung beschleunigt ($a = \SI{3}{\m\per\s\squared}$) auf eine andere, stehende zu ($v_0 = \SI{0}{\m\per\s}$).
Bei dem auftretenden unelastischen Stoß werden \SI{90}{\percent} der kinetischen Energie in Verformungsarbeit umgesetzt.
Berechnen Sie die Verformungsarbeit in Abhängigkeit vom Anlaufweg $s$ und stellen Sie den Zusammenhang grafisch dar.

\subsection[Elektrizitätslehre]{Elektrizitätslehre:}
\begin{enumerate}[leftmargin=0.5cm]
	\item Die Kuh beiße in den elektrisch geladenen Weidezaun ($U = \SI{40}{\V}$).
	Ein Strommessgerät registriert durch die Kuh einen Strom von \SI{0,5}{\mA}. Wie hoch ist der ohmsche Widerstand des Tieres?
	\item Dieselbe Kuh werde nun mit einer Spule ($L = \SI{0,5}{\henry}$) in Reihe geschaltet und an eine Wechselspannung von \SI{50}{\Hz} gelegt.
	Berechnen Sie den Scheinwiderstand $Z$ dieses RL-Gliedes und die Phasenverschiebung~$\varphi$ zwischen Strom und Spannung, wobei der Widerstand der Spule vernachlässigbar ist.
\end{enumerate}

\subsection[Quantenmechanik]{Quantenmechanik:}
\begin{enumerate}[leftmargin=0.5cm]
	\item Die Kuh befinde sich auf einer Weide, die ringsum durch einen Zaun abgegrenzt ist.
	Der Weidezaun sei ideal gebaut, sodass die Kuh ihn (klassisch gesehen) nicht passieren kann. Begründen Sie, dass man die Kuh trotzdem mit gewisser Wahrscheinlichkeit außerhalb der Weide antrifft!
	\item Unter Verletzung der Energieerhaltung können nach der Heisenbergschen~Unschärferelation kurzfristig sogenannte virtuelle Teilchen entstehen.
	Berechnen Sie die Lebensdauer einer virtuellen Kuh.
	\item "Schrödingers Kuh": Ein Mensch sperrt eine Kuh in einen Atombunker, aus dem keine Information nach außen dringt.
	Für den Beobachter ist die Kuh dann quantentheoretisch sowohl tot als auch lebendig (nicht "entweder\dots\ oder\dots"!).
	Erklären Sie den scheinbaren Widerspruch!
	\item Berechnen Sie die de~Broglie-Wellenlänge einer Kuh, die mit $v = \SI{10}{\m\per\s}$ auf der Weide galoppiert.
	Bis zu welchen Größenordnungen könnte man mit dieser Welle in der Mikroskopie Strukturen auflösen?
	Wieso benutzt man in der Strukturforschung keine Kühe?
\end{enumerate}

\subsection[Kernphysik]{Kernphysik:}
Die Kuh frisst auf der Weide 8~Stunden lang pro Stunde \SI{2}{\kg} radioaktiv verseuchtes Gras mit einem \ch{^{40}K}-Gehalt von \SI{0,01}{\percent}.
Während dieser Zeit scheidet die Kuh stündlich Fladen von \SI{1}{\kg} aus (die \ch{^{40}K}-Konzentration in den Fladen sei näherungsweise ebenfalls \SI{0,01}{\percent}).
Berechnen Sie die Anzahl der \ch{^{40}K}-Atome in der Kuh drei Wochen nach der Beendigung des Fressens unter Verwendung geeigneter Näherungen (die Kuh stelle während dieser Zeit auch das Abkoten ein).

\vspace{-0.6cm}
\fibelsig{Autor unbekannt}
