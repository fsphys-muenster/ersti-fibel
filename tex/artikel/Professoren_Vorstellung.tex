% XXX Jedes Jahr Professoren-Texte aktualisieren!
\section[Eure Profs stellen sich vor]{Eure Professoren stellen sich vor}
\textbf{Auf den folgenden zwei Seiten stellen sich eure beiden Professoren vor.
    Sie werden gemeinsam die "Physik~1" bis "Physik~3" lesen.
    Prof.\ Rohlfing wird sich dabei um die theoretischen und Prof.\ Wurstbauer um die experimentellen Aspekte des Studiums kümmern.
    Zudem stellt sich Prof.\ Werner vor, der die Vorlesungen "Mathematik für Studierende der Physik" halten wird (ebenfalls über drei Semester) und Dr. Kovarik, der sich in den ersten beiden Semestern um die Übungen kümmern wird.
	Da diese drei Professoren euch eine Zeit lang begleiten werden, ist es durchaus mal interessant zu wissen, was sie gemacht haben, bevor sie an die Uni Münster kamen, und wie ihre aktuelle Forschung aussieht.}

\begin{multicols}{2}
\begin{center}
\includegraphics[width=0.71\columnwidth]{res/vorstellungsfotos/rohlfing.jpg}\\
Prof.\ Dr.\ Michael Rohlfing\\
Institut für Festkörpertheorie
\end{center}

Herzlich willkommen an der Universität Münster, am Fachbereich Physik und in unserem integrierten Kurs Physik 1-3! Meine Kollegin Ursula Wurstbauer und ich werden Ihnen gemeinsam die Grundlagen der Physik vermitteln - in experimenteller und theoretischer Hinsicht gleichermaßen.

Es mag Sie interessieren, dass ich selber vor 34 Jahren mein erstes Semester in genau denselben Räumen hier in Münster erlebt habe. Nach meinem Diplom und der Promotion verbrachte ich einige Zeit an der University of California in Berkeley und kam dann nach Münster zurück, um hier zu habilitieren. Dann verschlug es mich 2003 als frisch berufener Professor weiter nach Norden, erst nach Bremen und dann nach Osnabrück. 2013 kehrte ich wiederum nach Münster zurück, auf meine jetzige Stelle am Institut für Festkörpertheorie. Ich kenne also den Fachbereich aus sämtlichen Perspektiven.

In der Forschung befassen sich meine Arbeitsgruppe und ich mit Festkörperphysik, insbesondere mit Elektronenstrukturtheorie. Wir wenden unsere Methoden auf alles an, was man als "kondensierte Materie" bezeichnet: Moleküle, Polymere, Adsorbate auf Oberflächen, nanoskalig strukturierte Systeme, und seit einigen Jahren verstärkt zweidimensionale Materialien wie Graphen oder atomar dünne Halbleiter, etwa Molybdänsulfid. Die Elektronen bilden die chemischen Bindungen zwischen den Atomen und kontrollieren dadurch den strukturellen Aufbau der Materie. Vor allem aber interessieren uns die optischen Spektren, und dafür brauchen wir ebenfalls die Elektronen: wenn Sie einen Farbstoff sehen, sehen Sie letztlich deren quantenmechanische Zustände.

Diese Quantenmechanik werden Sie in Physik 1-3 nur in Ansätzen kennenlernen, denn unser Schwerpunkt liegt zunächst auf der klassischen Physik: Mechanik, Elektrodynamik, Thermodynamik. Aber auch in der mikroskopischen Welt werden Sie diese Dinge später wiederfinden: die Bewegung der Atome in einem Molekül ist in weiten Teilen durch klassische Mechanik geprägt, die Coulomb-Wechselwirkung zwischen Atomkernen und Elektronen gehorcht den Regeln der Elektrodynamik, und die thermodynamischen Eigenschaften auf der Mikroskala genügen denselben Grundkonzepten wie ein Ottomotor oder eine Wärmepumpe. Daher vermitteln wir Ihnen drei Semester lang diese immer noch gültigen Grundlagen, bevor es ab dem vierten Semester verstärkt in die "moderne" Physik gehen wird.

Ich wünsche Ihnen ganz herzlich einen guten Start in Ihr Studium, viele gute neue Freunde, tolle Erfahrungen und Eindrücke, viel Erfolg und ganz viel Spaß an der Physik!


\end{multicols}

\begin{center}
    \fibelimgtext{
	\includegraphics[width=0.7\textwidth]{res/xkcd/895_teaching_physics.png}
    }{\url{https://xkcd.com/895}}
\end{center}

\newpage

\begin{multicols}{2}
\begin{center}
\includegraphics[width=0.8\columnwidth]{res/vorstellungsfotos/wurstbauer.jpg}\\
\smallskip
Prof.\ Dr.\ Ursula Wurstbauer\\
Physikalisches Institut
\end{center}

Auch von meiner Seite herzlich willkommen an der Universität Münster und insbesondere am Fachbereich Physik. Gleichzeitig möchte ich Ihnen zu Ihrer Entscheidung das Physikstudium hier bei uns aufzunehmen gratulieren. Sie dürfen im Laufe der kommenden Semester in die spannende, vielfältige Welt der Physik eintauchen.

Persönlich wollte ich zunächst Physik und Sport für das Lehramt an Gymnasien studieren, es war aber nur Physik und Mathematik möglich. So nahm ich zunächst dieses Studium an der Universität Regensburg auf. Schlüsselereignis, um neben dem Lehramts- auch den Physik-Diplomabschluss anzustreben, war ein Versuch im Fortgeschrittenenpraktikum zum Quanten-Hall-Effekt (Physik Nobelpreis 1986), bei dem wir selbst aus einem Stück Halbleiterkristall Mini-Drähte anlöteten, diese mit flüssigem Helium auf -269°C abkühlten und bei großen Magnetfeldern das Experiment durchführten. So motiviert von der Arbeit im Labor und fasziniert von den Eigenschaften der Festkörper habe ich, nach erstem Staatsexamen und Physikdiplom, an der Uni Regensburg promoviert und bin dann nach einem ersten Postdoc an der Universität Hamburg für mehrere Jahre an der Columbia Universität in New York City tätig gewesen. Nach der Rückkehr aus den USA leitete ich an der TU München eine Nachwuchsgruppe und habilitierte. Seit 2019 bin ich nun an der WWU als Professorin am Physikalischen Institut tätig. 

Inhaltlich beschäftigen wir uns experimentell mit Anregungen in Quantum-/ Nanosystemen – derzeit meist in zweidimensionalen Kristallen. Dieses vielfältige Gebiet reicht von grundlegenden Eigenschaften in neuartigen Materialien, die beispielsweise zur solaren Energiegewinnung geeignet sind, über deterministisch erzeuge Quantenemitter, dem Verhalten neuartiger Supraleiter und anderen Quantenmaterialien. Dazu betreiben wir eine Reihe von Optik- und Transportlabore, nutzen Reinräume, um opto-/elektronische Schaltkreise und Proben herzustellen und haben im Labor mit < -272.12°C (<10mK) den kältesten Ort in Münster. 

Vielleicht waren sie zuletzt von Freunden oder der Familie mit der Frage konfrontiert: „Physik-Studium – und was macht man dann damit?“ Die Antwort ist meist nur im Zweifachbachelor-Studium (Lehramt) klar. Forschung, Entwicklung, Beratung, Versicherungen, Banken, Halbleiterindustrie, Automotive, Ingenieurwesen, Softwarearchitekt, ... Das ist nur eine unvollständige Berufsliste, in denen meine Alumni und Kommilitonen tätig sind. Denn was Sie auf jeden Fall lernen ist neben viel Physik und dem entsprechenden „Handwerkszeug“, komplexe Zusammenhänge zu erkennen und zu beschreiben, sich neue Themen zu erarbeiten, vielfältige Problemlösestrategien entwickeln, Arbeit im Team, teils im internationalen Umfeld - und eine hohe Frustrationstoleranz. Aber keine Sorge, der Spaß kommt dabei auch nicht zu kurz. Und so werden Sie zum begehrten Generalisten, dem nach dem Studium sehr viele Türen offenen stehen - natürlich und wichtig auch der Weg in das Lehramt.

Ich wünsche Ihnen viel Erfolg, gute Freunde und Kommilitonen und viel Spaß im Studium – und uns allen eine tolle gemeinsame Zeit in den Physik 1-3 Vorlesungen. 

\begin{center}
\includegraphics[width=0.85\columnwidth]{private/res/comics/manchmal_edited.jpg}\\
{\footnotesize 
S.~Harris – \url{sciencecartoonsplus.com}
}
\end{center}

\end{multicols}

\vfill

\newpage

\begin{multicols}{2}
\begin{center}
\includegraphics[width=0.9\columnwidth]{res/vorstellungsfotos/wulkenhaar.png}\\
Prof.\ Dr.\ Raimar Wulkenhaar\\
Mathematisches Institut
\end{center}

Ich bin von der Ausbildung her Physiker, arbeite im Grenzgebiet zwischen Mathematik und Physik und bin seit 2005 Professor für Reine Mathematik am Fachbereich Mathematik und Informatik der WWU.

Die Vorlesung "Mathematik für Studierende der Physik" wird traditionell vom Mathematischen Institut veranstaltet. Ich selbst werde den Zyklus zum 9.~Mal halten. Es ist aus meiner Sicht eine schöne Vorlesung; mir ist aber klar, dass die meisten Studierenden das anders sehen.

Auch wenn der Stoff durchaus umfangreich ist, können wir nur einen kleinen Teil dessen behandeln, was die Physik benötigt. Es geht in der Vorlesung nicht um die Bereitstellung von Rechenwerkzeugen für die Physik; das bekommen Sie nebenbei in den Physikvorlesungen geliefert. Es geht in der Mathematik darum zu verstehen, weshalb diese Rechenwerkzeuge so und nicht anders funktionieren. Der Einstieg in die Denkweise der Mathematik ist für viele nicht leicht. Erst im Lauf der Zeit entsteht rückblickend ein gewisses Verständnis für die tiefliegenden Strukturen und Zusammenhänge der Mathematik. Im Idealfall gelangen Sie so zu einer soliden Grundlage, mit der Sie die Rechenwerkzeuge der Physik nicht nur verstehend nutzen, sondern kreativ weiterentwickeln können.


\[
\resizebox{0.45\hsize}{!}{$\displaystyle\sum_{n = 1}^\infty \frac{1}{n^2} = \frac{\pi^2}{6}$}
\]

Nun noch einige Informationen zu mir. Nach dem Physikstudium an der Universität Leipzig mit Abschluss als Diplomphysiker 1994 habe ich in Leipzig auch meine Doktorarbeit geschrieben und 1997 verteidigt. Dabei ging es um die Formulierung von Modellen der Teilchenphysik im Rahmen der nichtkommutativen Geometrie. Die Ergebnisse sind rückblickend völlig unwichtig, sie haben mich aber 1998/1999 als DAAD-Postdoc nach Marseille gebracht.

Ich habe am Centre de Physique Theorique in Marseille mein Arbeitsgebiet gefunden - die Quantenfeldtheorie auf nichtkommutativen Geometrien. Vereinfacht gesagt geht es um die Frage (und ihre Konsequenzen), ob man auf beliebig kleinen Längenskalen, sagen wir $10^{-80}$\,m, noch Physik betreiben kann. Es gibt gute Gründe anzunehmen, dass das unmöglich ist, und entsprechend sollte zur Formulierung physikalischer Gesetze eine Geometrie benutzt werden, in der $10^{-80}$\,m ebenfalls sinnlos sind. Diese Nichtkommutative Geometrie wird in einer Sprache analog zur Quantenphysik beschrieben.

Seit Marseille, vor allem aber seit meiner zweiten Postdoc-Station 2000/2001 an der Universität Wien, arbeite ich an quantenfeldtheoretischen Modellen auf einer besonders einfachen nichtkommutativen Geometrie. Während meines dritten Postdoc-Aufenthalts 2002/2005 am Max- Planck-Institut für Mathematik in den Naturwissenschaften in Leipzig konnte ich mit meinem Kollegen aus Wien zusammen eine größere Hürde beseitigen. Die mathematisch rigorose Konstruktion einer 4-dimensionalen Quantenfeldtheorie auf einer nichtkommutativen Geometrie konnten wir inzwischen abschließen. Etwas analoges ist in der üblichen kommutativen Geometrie bisher nicht geglückt. Es resultierten spannende Fragen, an denen ich seitdem arbeite.

\begin{center}
\includegraphics[width=0.85\columnwidth]{private/res/comics/calvin_mathe.pdf}
\end{center}
\end{multicols}
