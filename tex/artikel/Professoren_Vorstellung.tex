% XXX Jedes Jahr Professoren-Texte aktualisieren!
\section[Eure Profs stellen sich vor]{Eure Professoren stellen sich vor}
\textbf{Auf den folgenden zwei Seiten stellen sich eure beiden Professoren vor.
    Sie werden gemeinsam die "Physik~1" bis "Physik~3" lesen.
    Prof.\ Thiele sowie Prof.\ Wittkowski werden sich dabei um die theoretischen und Prof.\ Fallnich um die experimentellen Aspekte des Studiums kümmern.
    Zudem stellt sich Prof.\ Werner vor, der die Vorlesungen "Mathematik für Studierende der Physik" halten wird (ebenfalls über drei Semester).
	Da diese drei Professoren euch eine Zeit lang begleiten werden, ist es durchaus interessant zu wissen, was sie gemacht haben, bevor sie an die Uni Münster kamen, und wie ihre aktuelle Forschung aussieht.}

\begin{multicols}{2}
Liebe Erstsemesterstudentinnen und -studenten, fühlen Sie sich herzlich willkommen! Wir -- also Ihr Physik-Lehr-Team 2023 bis 2025 -- freuen uns, dass Sie eine gute Wahl getroffen und sich für ein Physikstudium an der Universität Münster entschieden haben. In den nächsten drei Semestern bieten wir Ihnen die Möglichkeit, sich regelmä{\ss}ig in den Vorlesungen Physik 1 bis 3 zu treffen, um sich durch die Themenfelder Mechanik, spezielle Relativitätstheorie, Thermodynamik sowie Elektrostatik und -dynamik bis hin zur Optik führen zu lassen. Dabei werden Prof.\ Fallnich für den experimentellen sowie Prof.\ Thiele und Prof.\ Wittkowski für den theoretischen Teil verantwortlich sein. Unsere Aufgabe ist es, Ihnen zu vermitteln, wie Sie sich die Physik und die zugehörigen inneren Zusammenhänge z.B. über Symmetriebetrachtungen systematisch erarbeiten können. Hierfür werden Ihnen die oben genannten Themenbereiche sowohl in Theorie als auch Experiment mit regelmä{\ss}igem wechselseitigem Bezug zueinander nähergebracht. Zur ersten Anwendung und auch Vertiefung Ihres erlernten Wissens wird Privatdozent Kovarik zusammen mit Übungs\-gruppen\-leiter\-innen und -leitern einen Übungsbetrieb umsetzen, durch den Sie anhand ausgewählter Beispielaufgaben die fachlichen Inhalte stärker im Detail kennenlernen werden. Für die Vorbereitung und Ausführung instruktiver Experimente in der Vorlesung ist Herr Horstmann zuständig.

Damit Sie wissen, mit welchen Dozenten Sie es direkt zu tun haben werden, möchten wir uns Ihnen gern kurz vorstellen:

Prof.~Raphael Wittkowski ist für den theoretischen Kursteil mitverantwortlich. Er arbeitet seit 2016 an der Universität Münster und forscht auf den Gebieten der Aktiven weichen Materie und der Statistischen Physik. Aktive weiche Materie besteht aus Teilchen, die sich selbstständig fortbewegen können, wie Mikroorganismen und Mikroroboter. Die Statistische Physik befasst sich mit der Beschreibung von Vielteilchensystemen. Seine Freizeit verbringt er gern in der Natur.

Prof.\ Uwe Thiele hat in den 1990ern Physik studiert, um herauszufinden, "was die Welt im Innersten zusammenhält". Nach einer ersten Tendenz in Richtung Elementarteilchenphysik, richtete sich sein Interesse aber schnell auf die Physik strukturbildender Systeme, an denen er auch heute noch mit ungebrochen großem Interesse forscht. Nach zahlreichen Forschungs- und Lehraufenthalten an verschiedenen Standorten im In- und Ausland verlagerte er vor ca.\ 10 Jahren seine Aktivitäten an die Universität Münster, wo seine Gruppe universelle Eigenschaften komplexer Nichtgleichgewichtssysteme mit analytischen und numerischen Methoden erforscht. Aufgrund vieler nichtlinear wechselwirkender Komponenten können Selbstorganisationsprozesse entstehen, die zur spontanen Entwicklung von räumlichen und zeitlichen Strukturen führen, die also nicht von au{\ss}en aufgeprägt werden. Im alltäglichen Leben zeigen sich solche Phänomene z.B. als Konvektionszellen im Milchkaffee, bei der Ausbildung von Mustern und Strukturen in der Tier- und Pflanzenwelt, und in der Dynamik von Wasserwellen, Sanddünen oder Wolkenbändern. Von gro{\ss}em aktuellem Interesse sind Tropfen komplexer Flüssigkeiten auf festen oder weichen Substraten, die Dynamik biologischer Zellen, die Entwicklung von Bakterienkolonien und allgemein die datengetriebene Modellierung.

\begin{center}
\includegraphics[width=0.8\columnwidth]{res/vorstellungsfotos/profs_ws23.jpg}\\
\smallskip
Gruppenfoto der Physik~1-3 Dozenten; in Ihrem Bezugssystem von links nach rechts: Carsten~Fallnich, Raphael~Wittkowski, Karol~Kovarik und Uwe~Thiele.
\end{center}

Prof.\ Carsten Fallnich wird sich um den experimentellen Kursteil kümmern und mit Experimenten und zugehörigen Erklärungen den theoretischen Vorlesungsstoff veranschaulichen und damit weitergehend vertiefen. Er kann auf mehr als drei Jahrzehnte Wissen und Erfahrung aus Forschung und Entwicklung zurückgreifen, davon zahlreiche auch außerhalb der Universität und seit 2006 in Münster. Er ist international anerkannter Experte in der Erzeugung und Anwendung von ultrakurzen Lichtimpulsen und der nichtlinearen Optik in Wellenleitern für die zukunftsweisende Photonik, welche nahezu alle Disziplinen der modernen Physik verbindet. Neben der Arbeit an der Universität verbringt Carsten Fallnich soweit wie möglich Zeit mit der Familie, joggt, fotografiert, fährt Zweiräder mit und ohne Motor und repariert (zumindest immer mit Erkenntnisgewinn!) aus Interesse an Nachhaltigkeit und Technikverständnis fast alles, was ihm ohne reguläre Funktion unter die Finger kommt.

Priv.-Doz.~Karol Kovarik organisiert den Übungsbetrieb in den Kursen Physik 1-3. Er bietet auch verschiedene freiwillige Angebote an, um Ihnen den Einstieg in die Physik zu erleichtern und um Ihnen die Gelegenheit zu geben, zusätzliche Fragen zu den Inhalten der Vorlesungen und übungen zu stellen. Seit seiner Promotion in 2006 forscht er im Feld der Elementar- und Astroteilchenphysik. Besonders spannend findet er die Suche nach der gro{\ss}en vereinheitlichten Theorie und Versuche, die innere Struktur des Protons zu verstehen. Wenn er nachts nicht schlafen kann, fotografiert er nächtliche Landschaften und den Sternenhimmel.

Das Verständnis komplexer Phänomene, welche das Leben in unserer Welt erst interessant gestalten und an denen z.B. viele  Arbeitsgruppen am Fachbereich Physik forschen, baut -- ob Sie es uns (schon jetzt) abnehmen mögen oder nicht -- auf den Inhalten der Grundvorlesungen auf, die nun auf Sie warten. Den notwendigen Stoffumfang dazu werden Sie sich mit Interesse, Motivation und Einsatz erfolgreich selbstständig unter unserer Anleitung aneignen können; dafür wünschen wir Ihnen Durchhaltevermögen, Freude und Erfolg. Denn es wird sich später für Sie in verschiedenster Weise auszahlen, in physikalischer Weise denken und handeln zu können. Sollten dann doch noch zwischendrin Fragen offen bleiben oder Sie sich bereits für Näheres zu unseren Forschungsarbeiten interessieren, kontaktieren Sie uns gerne unter den folgenden Kontaktdaten:\\

\begin{center}
\includegraphics[width=0.85\columnwidth]{private/res/comics/manchmal_edited.jpg}\\
{\footnotesize 
S.~Harris – \url{sciencecartoonsplus.com}
}
\end{center}

\end{multicols}

\vfill

\newpage

\begin{multicols}{2}
\begin{center}
	\includegraphics[width=\columnwidth, height=0.35\textheight]{res/vorstellungsfotos/wend_werner.jpg}\\
\smallskip
	Apl.\ Prof.\ Dr.\ Wend Werner\\
	Mathematisches Institut
\end{center}

Mir werden Sie in der nächsten Zeit in den Vorlesungen zur "Mathematik für Physiker" begegnen.
Mein Studium von Mathematik und Physik habe ich an der Freien Universität Berlin absolviert; Diplom und Promotion habe ich auch jeweils dort abgeschlossen.
Meine Habilitation habe ich an der Universität Paderborn gemacht und bin nun seit gut einem Jahrzehnt Hochschullehrer in Münster.

\[
	\resizebox{0.4\columnwidth}{!}{
		$\displaystyle\sum_{n = 1}^\infty \frac{1}{n^2} = \frac{\pi^2}{6}$
	}
\]

Die Themen von Promotion und Habilitation betrafen geometrische Fragen in Räumen unendlicher Dimension.
In letzter Zeit war das vor allem "Nichtkommutative Geometrie", ein Gebiet, welches versucht, einen mathematischen Formalismus zu finden, der in der Lage ist, Quanten- und relativistische Physik in einheitlicher Weise zu beschreiben.
Wie so oft beim Zusammenspiel von Mathematik und Physik sind auch hier interessante, rein mathematische Fragestellungen in Erscheinung getreten.

%\begin{center}
%	\includegraphics[width=\columnwidth, height=0.17\textheight]{private/res/comics/calvin_mathe.pdf}
%\end{center}

Der Zyklus "Mathematik für Physiker" ist eine kleine Herausforderung, da in vergleichsweise kurzer Zeit eine größere Stoffmenge vermittelt werden muss, die nichtsdestotrotz von den Teilnehmern anschließend handwerklich beherrscht werden muss.

Aber, keine Angst: Wir werden sehr langsam beginnen und erst im Laufe der Zeit Fahrt aufnehmen.
\end{multicols}

\begin{center}
	\fibelimgtext{
		\includegraphics[width=0.9\textwidth]{res/xkcd/435_purity.png}
	}{\url{https://xkcd.com/435}}
\end{center}
