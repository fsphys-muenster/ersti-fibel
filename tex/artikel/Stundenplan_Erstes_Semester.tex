\section[Stundenplan 1.~Semester]{Dein Stundenplan im 1.~Semester}
\vspace{-0.5cm}
\subsubsection*{für 1-Fach-Bachelor-Physik, 2-Fach-Bachelor-Physik und Geophysik}
\begin{minipage}{\textwidth}
% Keine Trennlinie vor Fußnoten in dieser minipage
\setfootnoterule{0cm}
% Die Länge \fibtemp ist die Breite einer Spalte in der Tabelle
% (außer Spalte mit den Zeiten)
\setlength{\fibtemp}{0.152\textwidth}
% Der Befehl \fibnl ist ein Zeilenumbruch
\let\fibnl=\par

\centering
% Verlinkung von Fußnoten im PDF klappt nicht mit tabularx :-(
% geht mit tabular, tabular*
\begin{tabular}{| >{\footnotesize}p{0.1\textwidth} | *{5}{>{\footnotesize\centering\arraybackslash}p{\fibtemp}|}}
\hline
	\textbf{Uhrzeit} &
	\textbf{Montag} &
	\textbf{Dienstag} &
	\textbf{Mittwoch} &
	\textbf{Donnerstag} &
	\textbf{Freitag}
\\ \hline
08:00--\fibnl
10:00 Uhr &
	\textbf{Physik~I\fibnl
		Übung}\footnote{Möglicher Termin. Es gibt weitere Termine für Übungsgruppen. Bei der Auswahl sollten die Termine mit den weiteren Vorlesungen, Übungen und Tutorien abgestimmt werden.\label{stundenplan:multi}}\fibnl
	 &
	\textbf{Mathe für Physiker~I\footnote{Nicht für 2-Fach-Bachelor-Studierende.\label{stundenplan:mfp1}} Übung\cref{stundenplan:multi}}
	 &
	\textbf{Physik~I Vorlesung}\fibnl
	HS~1 &
	\textbf{Physik~I\fibnl
		Übung\cref{stundenplan:multi}}\fibnl
	 &
	Informatik~I\cref{stundenplan:informatik} Übung\cref{stundenplan:multi}\fibnl
	
\\ \hline
10:00--\fibnl
12:00 Uhr &
	\textbf{Mathe für Physiker~I\cref{stundenplan:mfp1} Vorlesung}\fibnl
	KP~404 &
	\textbf{Physik~I Vorlesung}\fibnl HS~1 &
	&
	\textbf{Mathe für Physiker~I\cref{stundenplan:mfp1} Vorlesung}\fibnl
	KP~404 &
	\textbf{Physik~I Vorlesung}\fibnl
	HS~1
\\ \hline
12:00--\fibnl
14:00 Uhr &
	Chemie\footnote{Für 1-Fach-Bachelor Physik mit dem Modul Chemie als fachübergreifende Studien (Nebenfach). \textbf{Die Vorlesung Chemie ist nur 1-stündig von 12-13 Uhr.}.
	\label{stundenplan:chemie}} Vorlesung\fibnl
	C1 &
	Chemie\cref{stundenplan:chemie} Vorlesung\fibnl
	C1 \flushright
	&
	Chemie\cref{stundenplan:chemie} Vorlesung\fibnl
	C1 \fibnl
	\textbf{\& TUT (KP~404)}
	&
	Chemie\cref{stundenplan:chemie} Vorlesung\fibnl
	C1\fibnl
 	Einführung in die Geophysik\cref{stundenplan:geophysik} Übung (13-14 Uhr)\cref{stundenplan:multi}\fibnl
	GEO~315&
%\\ %\hdashline
%13--14 Uhr &
%	& & & &
\\ \hline
14:00--\fibnl
16:00 Uhr &
	Informatik~I\footnote{Für 1-Fach-Bachelor Physik mit dem Modul Informatik als fachübergreifende Studien (Nebenfach).
	\label{stundenplan:informatik}} Vorlesung\fibnl
	M~1+M~3 \fibnl
 	Einführung in die Geophysik\footnote{Für 1-Fach-Bachelor Geophysik sowie 1-Fach-Bachelor Physik mit dem Modul Geophysik als fachübergreifende Studien.
	\label{stundenplan:geophysik}} Vorlesung\fibnl
	HS~AP &
	&
	\textbf{REP (KP~404)}&
	Informatik~I\cref{stundenplan:informatik} Vorlesung\fibnl
	M~1+M~3 \fibnl
 	Chemie\cref{stundenplan:chemie} Übung\cref{stundenplan:multi} \fibnl
  	&
\\ \hline
16--18 Uhr &
	& 	\textbf{REP (KP~404)}&
	& &
\\ \hline
18:00--\fibnl
20:00 Uhr &
	&
	&
	Fachschafts-Sitzung\fibnl
	IG1 87\fibnl
	(Ihr seid alle willkommen!) &
	&
\\ \hline
\end{tabular}
\vspace{-1ex}
\end{minipage}
{\footnotesize
\textbf{REP} Mathe-Repetitorium: \textit{zweistündiges, freies Angebot zum Vertiefen grundlegender Mathematik mit wöchentl.\ Themen.}\\
\textbf{TUT} Tutorium: \textit{zweistündiges, freies Angebot zum Vertiefen allgemein grundlegender Themen.}
}

{\small
Des Weiteren können 1-Fach-Bachelor-Studierende, die weder Chemie, Geophysik, noch Informatik als fachübergreifende Studien wählen möchten, aus folgenden vorgefertigten Modulen als fachübergreifende Studien wählen:
\begin{itemize}[nosep]
	\item Philosophie für Studierende der Physik
	\item Mathematik
	\item Theoretische Grundlagen der Psychologie
	\item Einführung in die Betriebswirtschaftslehre
	\item Einführung in die Volkswirtschaftslehre
	\item Spanisch für Studierende der Naturwissenschaften
	\item (Deutsch als Fremdsprache)
\end{itemize}
oder ein selbst zusammengestelltes fachübergreifendes Modul, bei dem der Zusammenhang zur Physik erkennbar ist, oder das zur Berufsfähigkeit dient, bestimmen.
Dies ist aber nur nach Absprache mit dem Studiendekan möglich (aktuell: Prof.\ Dr.\ Hubert Krenner, \email{studiendekan.physik@uni-muenster.de}).
Die genauen Vorlesungs-/Übungszeiten der zusätzlichen fachübergreifenden Studien erfahrt ihr in der O-Woche von der Fachschaft Physik, direkt von der jeweiligen Fachschaft des zuständigen Fachbereichs oder ihr schaut einfach im Vorlesungsverzeichnis (HIS~LSF, \url{https://studium.uni-muenster.de/qisserver/rds?state=wtree&search=1}) der Uni nach.
}
