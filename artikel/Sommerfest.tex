% Autor: Simon May
% Datum: 2014-09-09

\section{Sommerfest 2014}
\begin{multicols*}{2}
% Etwas mehr Abstand zwischen Paragraphen, weil der Artikel so kurz ist
\setlength{\parskip}{2\parskip}

\textbf{Wieder einmal war es soweit, am 6.~Juni~2014 lud die Fachschaft~Physik ein, auf ihren Wiesen zu feiern, zu essen, zu trinken und zu tanzen.}

\includegraphicscompressed[width=\columnwidth]{res/sommerfest_foto_volleyball.png}
\medskip

Auch in diesem Jahr kam das Sommerfest der Fachschaft~Physik bei allen Beteiligten gut an und war ein großer Erfolg. Der Tag fing zunächst grau in grau an und das komplette Sommerfest war alles in allem recht "feucht". Insgesamt kamen ungefähr \num{1,2e2}~Leute (Zahl grob durch Autor geschätzt, kann von realen Gegebenheiten abweichen) und genossen das umfangreiche Rahmenprogramm bei Bier, Musik, Bratwürsten und Grillkäse.

{% LaTeX-Warnung deaktivieren
\hbadness=10000
% wrapfigure verwendet "\columnsep" für den Abstand zwischen Text und Bild 
\setlength{\columnsep}{0cm}
\begin{wrapfigure}{r}[0.5cm]{0cm}
\includegraphicscompressed[width=4.5cm]{res/sommerfest_foto_schuss.png}
\end{wrapfigure}
Pünktlich um 11:30~Uhr wurde der Grill angezündet und so konnten die ersten Besucher als Mittagessen eine gute Bratwurst genießen.

Durch eine Fragestunde zum Erasmus-Programm mit ehemaligen Erasmus-Studenten und dem Erasmus-Beauftragten, Prof.\ Dr.\ Kohl, hatte der Tag auch eine fast schon bildende Komponente.

Der etwas aufgeweichte Boden ließ die Teams des Volleyballturniers nicht davon abhalten, das lang ersehnte Spiel um 14:00~Uhr auszuführen, und so trafen die Mannschaften im K.O.-System aufeinander und boten sich einen harten, aber fairen Kampf um den Sieg. Von der Vorgruppe über Viertel- und Halbfinale bis hin zum Finale konnte jedes Team sein Können unter Beweis stellen.}

Auch in diesem Jahr wurde wieder ein Speed-Cubing-Wettbewerb veranstaltet, bei dem die Teilnehmer unter Beweis gestellt haben, dass sie den Zauberwürfel mühelos lösen können.

Um 17:00~Uhr wurde das von vielen herbeigefieberte Physiker-Duell veranstaltet. Bei diesem nach dem Prinzip des alten "Familien-Duell" konzipierten Spiel trafen vier Arbeitsgruppen des Fachbereiches Physik aufeinander und stellten sich den Fragen des Moderators Dennis.

Zuvor wurden "100~Physiker" gefragt und es galt natürlich, die meisten Antworten zu erraten. In diesem Jahr wurde im Finale die AG von Prof.\ Kohl von der AG~Thiele klar geschlagen, während die AGs Rohlfing und Bratschitsch in den Vorrunden ausschieden. 

Unterlegt wurde der ganze Tag von Musik, die mit einem breiten Spektrum an musikalischen Meisterwerken eine perfekte Kulisse schuf.

Da der Abend nach hinten hin offen gehalten wurde, blieben viele noch bis zum bitteren Ende um ca.~22:00 Uhr und beteiligten sich sogar noch an den Aufräumarbeiten.

Ein langes und erfolgreiches Sommerfest war nun beendet und sucht im nächsten Jahr einen würdigen Nachfolger. Dann natürlich auch mit euch, die ihr jetzt schon mal herzlichst eingeladen seid. Besonders im Bereich des Volleyballturniers werden starke Gegner gesucht, um die schon seit Jahren dominierenden Gruppen endlich von der Spitze ihres Ruhmes zu verdrängen.

\fibelsig{Bernd, Andreas GPunkt}
\end{multicols*}
