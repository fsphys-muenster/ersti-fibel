% Autor: Simon May
% Datum: 2014-08-19

% Textbereich dieser Seite 2 Zeilen größer machen, sonst wird aus irgendeinem
% Grund eine leere Seite eingefügt
\enlargethispage{2\baselineskip}
\section{Studieren im Ausland -- Das Erasmus-Programm}
\begin{pullquote}{shape=rectangular, objdist=0cm, objvoffset=2,
object={\includegraphics[width=4.5cm]{res/erasmus.jpg}}}
Erasmus? Ist das nicht ein irgendein Humanist? Erasmus von Rotterdam war ein niederländischer Theologe und Philosoph aus dem 15.~Jahrhundert, der als Vorreiter des Humanismus in die Geschichtsbücher eingegangen ist.

\hspace{1em}\textit{"Schön und gut. Aber was hat das mit mir zu tun?"}

fragst du dich sicherlich. ERASMUS hat noch eine andere Bedeutung: "\textbf{E}u\textbf{R}opean (Community) \textbf{A}ction \textbf{S}cheme for the \textbf{M}obility of \textbf{U}niversity \textbf{S}tudents". Dieses Programm ermöglicht es dir, überall in Europa relativ unbürokratisch zu studieren.

\hspace{1em}\textit{"Aber wieso sollte ich ins Ausland gehen?!"}

Das ist eine gute Frage. Die Antwort darauf kannst du dir letzten Endes nur selbst geben. Aber es sprechen viele gute Gründe dafür, einmal über den Tellerrand hinaus zu blicken und fremdes Terrain zu erkunden.\pullquotenl

Als erstes denkt man vermutlich an die fremde Sprache. Das ist aber seltsamerweise kein Problem. Obwohl ich beispielsweise bei meiner Ankunft in Lund kein einziges Wort Schwedisch sprechen konnte, konnte ich wunderbar auf Englisch kommunizieren. Auch den (englischen) Vorlesungen konnte ich problemlos folgen. Physik ist nun einmal universell und die Formelzeichen einheitlich! Zudem gab es dort kostenlose Sprachkurse an der Uni, so dass ich doch nicht ganz unwissend wieder zurückkam :)\pullquotenl

Als nächstes könnte man einwenden, dass die Vorlesungen vielleicht nicht zu dem passen, was man machen möchte und muss. Aber dieser Punkt darf getrost vergessen werden, da der Fachbereich~Physik nur Kooperationen mit den Universitäten eingeht, bei denen der Studienverlaufsplan kompatibel mit unserem ist. Dies ist der Hauptgrund dafür, statt einem Semester ein ganzes Jahr zu bleiben, damit man Quantenmechanik z.B. nicht doppelt hört. Deshalb raten viele, im 3.~Studienjahr zu gehen, damit man neben der QM auch die Statistische Physik abgedeckt hat. Man kann jedoch auch, so wie ich, im 7./8.~Semester reisen :)\pullquotenl

Weitere Bedenken kommen meist, wenn es um die Finanzierung geht. Schweden beispielsweise ist in der Tat kein günstiges Land, Lebensmittel sind im Schnitt \SIrange{30}{70}{\percent} teurer als hier. Wenngleich die Mietpreise vergleichbar bzw. leicht höher liegen als in Münster, so muss man mit ca.~800~Euro im Monat kalkulieren, um locker über die Runden zu kommen. Falls man keine reichen Eltern hat, ist das trotzdem kein Grund, auf das Ausland zu verzichten: Die Förderung über das International Office in Münster ist eine Finanzierungsquelle, außerdem besteht Anspruch auf Auslands-BAföG, wobei der Satz höher liegt als beim Inlands-BAföG. Falls ihr also hier keins bekommt, ist es trotzdem möglich, dass der Aufenthalt gefördert werden kann.\pullquotenl

Aber den Hauptgrund, wieso es sich lohnt, kann ich nicht in einem Artikel beschreiben. Man muss dieses Gefühl einmal selbst erlebt haben. Wenn man sich unbekümmert mit benachbarten Indern über die Bedeutung von Familie unterhalten, ein typisch australisches Frühstück verzehren, mit wildfremden Kroaten zusammen Bier trinken, mit (fast) nackten Schweden auf Tischen tanzen, mit betrunkenen Franzosen über das eigene Land herziehen und bis nachts um 5~Uhr zutiefst philosophische und religiöse Themen mit Ungarinnen ausdiskutieren kann, dann weiß man, dass man im Ausland studiert hat. Dieses unbändige Gefühl der Freiheit gepaart mit jugendlichem Idealismus und die Gewissheit, dass einem die Türen der Welt offen stehen, erlebt man vermutlich wirklich nur einmal im Leben bei genau dieser Gelegenheit.\pullquotenl

ERASMUS ist mehr als ein Eintrag im Lebenslauf. ERASMUS ist ein Stück Lebenserfahrung, das du nicht mehr missen möchtest. Alles ist anders. Überall fällt dir etwas auf, das ein kleines bisschen anders ist, als man es aus Deutschland kennt. Und wenn du erst einmal krank geworden bist, wirst du dankbar dafür sein, dass du in Deutschland krankenversichert sein musst. Du fängst an, darüber nachzudenken, welche Unterschiede es gibt. Welche Dinge hierzulande besser sind, welche dort. Du lernst nicht nur das Land besser kennen, sondern auch dein Heimatland. Und entgegen aller Sorgen und Ängste, die ich hatte, kann ich wirklich jedem wärmstens empfehlen, über seinen eigenen Schatten zu springen und das Wagnis "Ausland" einzugehen.\pullquotenl

Falls euer Interesse geweckt wurde, könnt ihr auf die Informationsveranstaltung im 2.~Semester warten, oder euch vorher bei uns bzw.\ im Internet erkundigen:

% URL wurde mit \href gemacht, um den Zeilenumbruch manuell einfügen zu können
\href{https://www.uni-muenster.de/Physik/Studieren/Auslandsstudium}{\textbf{\texttt{https://www.uni-muenster.de/Physik/\\Studieren/Auslandsstudium}}}

\fibelsig{Anna}
\end{pullquote}
