% Autor: Simon May
% Datum: 2014-09-22
\section{Geheimtipps für einen netten Abend}

\begin{multicols}{2}
\subsection*{Schnabulenz -- Open Mic}
\textbf{Adresse: Geiststraße 50}

\begin{center}
\includegraphicscompressed[width=\columnwidth]{res/foto_schnabulenz.jpg}
\end{center}

In der Schnabulenz findet wöchentlich mittwochs ab 19:00~Uhr ein Open Mic statt (ihr könnt ruhig eher da sein). Bei gemütlicher Wohnzimmeratmosphäre im Keller könnt ihr in fast familiärer Runde (max.\ 40~Leute) Freiwilligen beim Performen zuschauen.

Ob Musik, Comedy oder Lesung, man weiß nie was einen erwartet. Diese Wundertüte macht eine Empfehlung für ein erstes Date schwierig. Man kann Glück haben und bei feinster Comedy Tränen lachen, leidenschaftlichen Musikern lauschen -- oder sich bei einer Lesung über die Tschernobyl-Katastrophe fragen, ob man das Date noch `rum kriegt.

Außerdem ist das Ganze eher passiv, weil bei dem gebotenen Programm kein Gespräch zustande kommen kann. Daher lieber ein zweites oder drittes Date hier versuchen, um die Geschmäcker des anderen in verschiedensten Genres kennen zu lernen, und dann den Abend woanders ausklingen lassen.

Mutige können auch selbst auf die Bühne treten, müssen sich dafür nur kurz beim Veranstalter melden. Im Erdgeschoss gibt es außerdem eine kleine Küche und weitere Sitzgelegenheiten. Bier (\SI{0,3}{\l}) kostet \SI{2,60}{€}.

\vspace{\fill}

\columnbreak

\subsection*{Pension Schmidt -- Students' Battle}
\textbf{Adresse: Alter Steinweg 37}

\begin{center}
\vspace{-1.2mm}
\includegraphicscompressed[width=\columnwidth]{res/foto_pension_schmidt.jpg}
\end{center}

In der Pension Schmidt findet alle zwei Wochen (in der Regel am ersten und dritten Donnerstag im Monat) ein Students' Battle statt. Hierbei treten etwa 20~Teams aus 5--7~Personen gegeneinander an.

Das Quiz startet um 20:30~Uhr, allerdings solltet ihr etwa um 19:00~Uhr da sein, um noch einen Tisch zu bekommen. Es werden nacheinander drei Runden gespielt, in denen nicht nur Fragen zum Tagesgeschehen, Latein, Geographie oder Naturwissenschaften gestellt werden, sondern ihr euch auch in Aktionsspielen (Papierflieger bauen, Zahnpasta schmecken\dots) mit anderen Studenten messen könnt. Zwischen den Runden bestellt ihr am besten neue Getränke, um keine Frage zu verpassen.

Die Pension Schmidt bietet urige Wohnzimmeratmosphäre wie bei den Großeltern. Wo sonst ist es möglich, dass einen heiße Schokolade (naja, eigentlich heiße Milch mit echter Schokolade zum selbst Zusammenrühren!!!) neben einem Bier getrunken wird?
\vspace{1.6cm}
\fibelsig{Axel, Judith, Maik}

\end{multicols}
\begin{center}
\includegraphics[width = 0.6\textwidth]{./res/comics/xkcd_responsible_behavior.png}
\end{center}

