\documentclass[12pt,a4paper,twocolumn]{article}
\usepackage[utf8]{inputenc}
\usepackage[german]{babel}
\usepackage{a4}


\begin{document}

{\large\noindent Prof.\ Dr.\ Raimar Wulkenhaar\\
Mathematisches Institut (FB 10)}

\bigskip

\noindent
Ich bin von der Ausbildung her Physiker, arbeite im Grenzgebiet
zwischen Mathematik und Physik und bin seit 2005 Professor für Reine
Mathematik am Fachbereich Mathematik und Informatik der WWU. 

Die Vorlesung ``Mathematik für Physiker'' wird traditionell vom
Mathematischen Institut veranstaltet. Ich selbst werde den Zyklus zum
5. Mal halten.  Es ist aus meiner Sicht eine schöne Vorlesung; mir ist
aber klar, daß die meisten Studierenden das anders sehen. Auch wenn
der Stoff durchaus umfangreich ist, können wir nur einen kleinen Teil
dessen behandeln, was die Physik benötigt. Es geht in der Vorlesung
nicht um die Bereitstellung von Rechenwerkzeugen für die Physik; das
bekommen Sie nebenbei in den Physikvorlesungen geliefert. Es geht in
der Mathematik darum \emph{zu verstehen, weshalb} diese
Rechenwerkzeuge so und nicht anders funktionieren. Der Einstieg in die
Denkweise der Mathematik ist für viele nicht leicht. Erst im Lauf der
Zeit entsteht rückblickend ein gewisses Verständnis für die
tief\/liegenden Strukturen und Zusammenhänge der Mathematik. Im
Idealfall gelangen Sie so zu einer soliden Grundlage, mit der Sie die
Rechenwerkzeuge der Physik nicht nur verstehend nutzen, sondern
kreativ weiterentwickeln können.

Nun noch einige Informationen zu mir. Nach Physikistudium an der
Universität Leipzig mit Abschluß als Diplomphysiker 1994 habe ich in
Leipzig auch meine Doktorarbeit geschrieben und 1997 verteidigt. Dabei
ging es um die Formulierung von Modellen der Teilchenphysik im Rahmen
der nichtkommutativen Geometrie. Die Ergebnisse sind rückblickend
völlig unwichtig, sie haben mich aber 1998--1999 als DAAD-Postdoc nach
Marseille gebracht.  Ich habe am Centre de Physique Théorique in
Marseille mein Arbeitsgebiet gefunden, die Quantenfeldtheorie auf
nichtkommutativen Geometrien. Vereinfacht gesagt geht es um die Frage
(und ihre Konsequenzen), ob man auf beliebig kleinen Längenskalen,
sagen wir $10^{-80}\,\mathrm{m}$, noch Physik betreiben kann. Es gibt
gute Gründe anzunehmen, daß das unmöglich ist, und entsprechend sollte
zur Formulierung physikalischer Gesetze eine Geometrie benutzt werden,
in der $10^{-80}\,\mathrm{m}$ ebenfalls sinnlos sind. Diese
Nichtkommutative Geometrie wird in einer Spache analog zur
Quantenphysik beschrieben. Seit Marseille, vor allem aber seit meiner
zweiten Postdoc-Station 2000--2001 an der Universität Wien, arbeite
ich an quantenfeldtheoretischen Modellen auf einer besonders einfachen
nichtkommutativen Geometrie. Während meines dritten
Postdoc-Aufenthalts 2002--2005 am Max-Planck-Institut für Mathematik
in den Naturwissenschaften in Leipzig konnte ich mit meinem Kollegen
aus Wien zusammen eine größere Hürde beseitigen. Die mathematisch
rigorose Konstruktion einer 4-dimensionalen Quantenfeldtheorie auf
einer nichtkommutativen Geometrie konnten wir vor wenigen Monaten in
Münster/\mbox{Wien} abschließen. Etwas analoges ist in der üblichen
kommutativen Geometrie bisher nicht geglückt.







\end{document}

%%% Local Variables: 
%%% mode: latex
%%% TeX-master: t
%%% End: 
