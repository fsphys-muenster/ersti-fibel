\documentclass[12pt,a4paper,english]{article}
\usepackage[english]{babel}


\begin{document}
\Large
Bereits w\"ahrend meiner Schulzeit war mir klar: Meine berufliche
Zukunft liegt auf dem Gebiet der Naturwissenschaften. Mathematik
und Physik fiel mir leicht, Chemie schon etwas schwerer. 
Deshalb studierte ich von 1977-1982  
Physik an der Universit\"at Stuttgart. Zu dieser Zeit wurde gerade 
ein neues Forschungsgebiet aktuell, das unter dem Schlagwort
{\em Chaostheorie} auf junge, theoretisch orientierte Physiker
eine ungeheuere Faszination aus\"ubte. Da es in Stuttgart 
einen weltweit bekannten Lehrstuhl genau mit dieser Forschungsrichtung
gab, war ich froh, dort eine Diplomarbeit anfertigen zu k\"onnen. 
Nach Ableistung meines 
Zivildienstes kam ich 1983 an dieses Institut zur\"uck und promovierte
\"uber {\em Station\"are, wellenartige und chaotische Konvektion in
Systemen mit Kugelgeometrie}. Diese Problemstellung war von
geophysikalischen Fragestellungen wie z.B. der Konvektion im Erdmantel
oder der Frage nach der Erzeugung des Erdmagnetfeldes
motiviert. Im Anschluss an meine Promotion war ich direkt in die
Forschungsaktivit\"aten des Institutes eingebunden. Diese
bewegten sich dabei immer st\"arker auf die
Untersuchung dynamischer und chaotischer Vorg\"ange in Biologie und Medizin
zu. Wir lernten interdisziplin\"ar zu arbeiten und dabei
unsere Kenntnisse der statistischen Physik und der Theorie dynamischer
Systeme auf Problemstellungen wie etwa die Analyse von Gehirnsignalen
amzuwenden. Diese Arbeiten f\"uhrten 1992 zu meiner Habilitation. Die 
Habilitationsschrift tr\"agt den Titel {\bf Dynamische Strukturen in 
synergetischen Systemen}. Im Jahre 2001 nahm ich einen Ruf an das
Institut f\"ur Theoretische Physik der WWU M\"unster an. 
Meine Arbeitsgruppe besch\"aftigt sich mit der Theorie komplexer
Systeme. Gegenstand sind Selbstorganisationsprozesse in
Nichtgleichgewichtssystemen, die zur Bildung r\"aumlicher und
raumzeitlicher Strukturen Anlass geben. 
Die
Schwerpunkte meiner Arbeitsgruppe sind dabei {\em Strukturbildung in
Nichtgleichgewichtssystemen}, die {\em Analyse komplexer Systeme} sowie die
Behandlung {\em turbulenter Systeme}. 
Die Faszination, die diese Probleme
auf mich aus\"uben und die ich mit StudentInnen und DoktorandInnen
teilen m\"ochte, ist nach wie vor ungebrochen. 
Die Erforschung der Eigenschaften komplexer
Nichtgleichgewichtssysteme hat eigentlich erst begonnen.             
     


\end{document}





