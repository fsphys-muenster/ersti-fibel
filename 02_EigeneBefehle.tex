% Autor: Simon May
% Datum: 2014-09-22

% Variablen für Titelseite/Seitenvorlage
% Jahr und Semester gegebenenfalls ändern!
\newcommand{\fibeltitel}{Erstsemester-Informationszeitung des Fachbereichs Physik der Westfälischen Wilhelms-Universität Münster}
\newcommand{\fibeljahr}{2014/2015}
\newcommand{\fibelsemester}{Wintersemester}
\newcommand{\fibelautor}{Fachschaft Physik der WWU Münster, Simon May}

% Befehl "\fibelsig": Für Unterschriften unter Artikeln
% 	Parameter #1: Name des Autors
\newcommand{\fibelsig}[1]{\par\hfill(#1)\par}

% Befehl "\fibelwerbung": Für eine Seite mit Werbung
% 	Optionaler Parameter #1: Erstellt eine Seite mit dem Parameter als Inhalt;
%	falls leer: Verwendet das Wort "Werbung" in groß und zentriert
\NewDocumentCommand{\fibelwerbung}{O{\centering\vspace*{10cm}\Huge\bfseries Werbung\par}}
{\begin{center}#1\end{center}\clearpage}

% Befehl "\email": E-Mail-Adresse mit mailto-Link (anklickbar in der PDF-Datei)
% 	Parameter #1: Gewünscht E-Mail-Adresse
\newcommand{\email}[1]{\href{mailto:#1}{\texttt{#1}}}

% Befehl "\includegraphicscompressed": Wenn es einen Ordner "compressed" mit
% einer Datei der Form (z.B.) <name>_scaled.jpg gibt, wird diese statt der
% angegebenen Datei eingebunden.
% Befehl "\includegraphicscompressed": Wenn es einen Ordner "compressed" mit
% einer Datei der Form <name>_scaled.jpg oder <name>.jpg gibt, wird diese statt
% der angegebenen Datei eingebunden.
\newcommand{\includegraphicscompressed}[2][]{%
\reversestring[e]{#2}%
\let\rstring\thestring%
\whereisword[q]{\rstring}{.}%
\let\rdotpos\theresult%
\whereisword[q]{\rstring}{/}%
\let\rslashpos\theresult%
\stringlength[q]{#2}%
\let\len\theresult%
\def\dotpos{\numexpr\len - \rdotpos + 1}%
\def\slashpos{\numexpr\len - \rslashpos + 1}%
%
\substring[q]{#2}{1}{\slashpos}%
\let\temp\thestring%
\substring[q]{#2}{\slashpos}{\numexpr\dotpos-1}%
\def\namenoextension{\temp compressed\thestring}%
%
\IfFileExists{\namenoextension_scaled.jpg}%
{\includegraphics[#1]{\namenoextension_scaled.jpg}}%
{\IfFileExists{\namenoextension.jpg}%
{\includegraphics[#1]{\namenoextension.jpg}}%
{\includegraphics[#1]{#2}}}%
}

% Zähler für Umgebung fibelurl (wird mit jeder \section zurückgesetzt)
\newcounter{fibelurlctr}[section]
\crefformat{fibelurlctr}{#2[#1]#3}

% Umgebung "fibelurl": Für nummerierte Linklisten mit Verweisen wie z.B. im
% Artikel Geld/BAföG
\newenvironment{fibelurl}
{\refstepcounter{fibelurlctr}[\arabic{fibelurlctr}]~}
{\par}

% Befehl "\pullquotenl": Leerzeile für die pullquote-Umgebung (Bild in der Mitte,
% von Text umflossen)
\newcommand{\pullquotenl}{\par~}

%%%%%%%%%%%%%%%%%%%%%%%%%%%%%%
% !!! Die Befehle ab hier sind nicht zur direkten Verwendung durch den gemeinen
% !!! Fibel-Schreiberling gedacht ;-)

% Temporäre Variable
\newlength{\temp}

% Definition der Linie in der Kopfzeile
\NewDocumentCommand{\fibelheadsepline}{O{even}}
{\vspace{-0.6cm}%
\ifthenelse{\equal{#1}{even}}{\Huge$\Phi$\hfill}{}%
\rule[2.6mm]{0.96\linewidth}{0.5mm}%
\ifthenelse{\equal{#1}{odd}}{\hfill\Huge$\Phi$}{}}

% Definition des Textes an den Seitenrändern jeder Seite
\newcommand{\fibelmargin}{\Large Ersti-$\Phi$bel \fibeljahr}


%%%%%%%%%%%%%%%%%%%%%%%%%%%%%%
% Mathe-Definitionen
\renewcommand{\Re}{\operatorname{Re}}
\renewcommand{\Im}{\operatorname{Im}}

\newcommand{\NN}{\mathbb{N}}
\newcommand{\ZZ}{\mathbb{Z}}
\newcommand{\QQ}{\mathbb{Q}}
\newcommand{\RR}{\mathbb{R}}
\newcommand{\CC}{\mathbb{C}}
\newcommand{\abs}[1]{\left\lvert #1 \right\rvert}
\newcommand{\norm}[1]{\left\lVert#1\right\rVert}
\newcommand{\arsinh}{\operatorname{arsinh}}
\newcommand{\arcosh}{\operatorname{arcosh}}
\newcommand{\artanh}{\operatorname{artanh}}

\makeatletter
% Definition eines aufrechten "d" für Integrale (z.B. "\int x^2 \diff x")
\newcommand*{\diff}%
		{\@ifnextchar^{\DIfF}{\DIfF^{}}}
\def\DIfF^#1{%
		\mathop{\mathrm{\mathstrut d}}%
			\nolimits^{#1}\gobblespace
}
\def\gobblespace{%
        \futurelet\diffarg\opspace}
\def\opspace{%
        \let\DiffSpace\!%
        \ifx\diffarg(%
                \let\DiffSpace\relax
        \else
                \ifx\diffarg[%
                        \let\DiffSpace\relax
                \else
                        \ifx\diffarg\{%
                                \let\DiffSpace\relax
                        \fi\fi\fi\DiffSpace}

\makeatother

\newcommand*{\pderiv}[3][]{\frac{\partial^{#1}#2}{\partial #3^{#1}}}
\newcommand*{\deriv}[3][]{\frac{\diff^{#1}#2}{\diff #3^{#1}}}

\NewDocumentCommand{\ceil}{s O{} m}{%
	\IfBooleanTF{#1} % starred
		{#2\lceil#3#2\rceil} % \ceil*[..]{..}
		{\left\lceil#3\right\rceil} % \ceil[..]{..}
}
\NewDocumentCommand{\floor}{s O{} m}{%
	\IfBooleanTF{#1} % starred
		{#2\lfloor#3#2\rfloor} % \floor*[..]{..}
		{\left\lfloor#3\right\rfloor} % \floor[..]{..}
}

% Redefinitionen
\let\originalleft\left
\let\originalright\right
\renewcommand{\left}{\mathopen{}\mathclose\bgroup\originalleft}
\renewcommand{\right}{\aftergroup\egroup\originalright}

